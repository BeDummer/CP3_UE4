\documentclass[%
	paper=A4,	% stellt auf A4-Papier
	pagesize,	% gibt Papiergröße weiter
	DIV=calc,	% errechnet Satzspiegel
	smallheadings,	% kleinere Überschriften
	ngerman		% neue Rechtschreibung
]{scrartcl}
\usepackage{BenMathTemplate}
\usepackage{BenTextTemplate}

\title{{\bf Wissenschaftliches Rechnen III / CP III}\\Übungsblatt 4}
\author{Tizia Kaplan (545978)\\Benjamin Dummer (532716)}
\date{25.05.2016}

\begin{document}
\maketitle
Online-Version: \href{https://www.github.com/BeDummer/CP3_UE4}{\url{https://www.github.com/BeDummer/CP3_UE4}}

\section*{Aufgabe 4.i}
Hier sollten die Laufzeiten fuer $n=2^24$ Elemente fuer unterschiedliche Blockgroessen mithilfe von \texttt{nvprof ./a.out} gemessen werden. Vergleichend wurden die elapsed Zeiten fuer die Implementierung \texttt{reduceInterleaved} und \texttt{reducedUnrolling} aufgenommen. Gefunden wurde ein Minimum bei Blocksize = 128 (siehe Abb. 1) 

\section*{Aufgabe 4.ii}
Nun konnten auch Global Load Throughput, Global Load Efficiency und Achieved Occupancy gemessen werden. Hier wurden mittels \texttt{nvprof --metrics gld_efficiency,gld_throughput,achieved_occupancy ./a.out} die Durchschnittswerte ermittelt.
ABBILDUNGEN

\section*{Anhänge}
\begin{itemize}
	\item Datei: \url{reduction.cu} (Hauptprogramm)
\end{itemize}
\end{document}
